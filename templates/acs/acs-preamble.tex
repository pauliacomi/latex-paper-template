% !TEX root = ../../manuscript.tex

%%%%%%%%%%%%%%%%%%%%%%%%%%%%%%%%%%%%%%%%%%%%%%%%%%%%%%%%%%%%%%%%%%%%%
%%%%%%%%%%%%%%%%%%%%%%%%%%%%%%%%%%%%%%%%%%%%%%%%%%%%%%%%%%%%%%%%%%%%%
%% In an ACS manuscript the preamble contains
%%      - extra packages imported
%%      - paper title
%%      - paper metadata
%%      - paper keywords
%%%%%%%%%%%%%%%%%%%%%%%%%%%%%%%%%%%%%%%%%%%%%%%%%%%%%%%%%%%%%%%%%%%%%
%%%%%%%%%%%%%%%%%%%%%%%%%%%%%%%%%%%%%%%%%%%%%%%%%%%%%%%%%%%%%%%%%%%%%

%%%%%%%%%%%%%%%%%%%%%%%%%%%%%%%%%%%%%%%%%%%%%%%%%%%%%%%%%%%%%%%%%%%%%
%% Latex preamble
%%%%%%%%%%%%%%%%%%%%%%%%%%%%%%%%%%%%%%%%%%%%%%%%%%%%%%%%%%%%%%%%%%%%%

%%%%%%%%%%%%%%%%%%%%%% Fonts and language

% UNICODE recognition:
% This package should recognise any UNICODE characters in the text and automatically replace them with their standard macros
\usepackage[utf8]{inputenc}

% Font encodings
\usepackage[TS1,T1]{fontenc}

% AMS maths packages
\usepackage{
    amsmath,
    amsfonts,
    amsthm,
    amssymb
}

%%%%%%%%%%%%%%%%%%%%%% Graphics packages, tables and listings

% Graphicx:
\usepackage{graphicx}

% Floats
\usepackage{float}

% Color definitions
\usepackage{xcolor}

% For the floatbarrier macro
\usepackage[section]{placeins}

% Better tables
\usepackage{
    booktabs,
    makecell,
    array,
    multirow,
    tabularx
}

% Better captions 
\usepackage{caption}

% Subfigures
\usepackage{subcaption}

% For code snippets
\usepackage{listings}

% Todo notes
% Used for notes and annotations.
\usepackage{todonotes}

%%%%%%%%%%%%%%%%%%%%%% Science-related packages

% In-line fractions
\usepackage{xfrac}

% The SIunitx package enables the \SI{}{} command.
\usepackage{siunitx}
\sisetup{detect-weight=true, detect-family=true}

% The mchem package for formula subscripts using \ce{}
\usepackage[version=4]{mhchem}

%%%%%%%%%%%%%%%%%%%%%% Bibliography packages and settings

%%%%%%%%%%%%%%%%%%%%%% References and bookmarks

% Xr:
% To reference another document, in this case the SI
\usepackage{xr, xr-hyper}

%% Hyperlinking
\usepackage{hyperref}
\hypersetup{colorlinks=false, pdfborder=0 0 0}

% Additional Setup
%%%%%%%%%%%%%%%%%%%%%%%%%%%%%%%%%%%%%%%%%%%%%%%%%%%%%%%%%%%%%%%%%%%%%

%% Lorem ipsum
\usepackage{lipsum}

%% References
\usepackage[
  capitalise,
  nameinlink,
]{cleveref}
\creflabelformat{equation}{#2\textup{#1}#3}


%%%%%%%%%%%%%%%%%%%%%%%%%%%%%%%%%%%%%%%%%%%%%%%%%%%%%%%%%%%%%%%%%%%%%
%% Title
%% -----
%% The document title should be given as usual. Some journals require
%% a running title from the author: this should be supplied as an
%% optional argument to \title.
%%%%%%%%%%%%%%%%%%%%%%%%%%%%%%%%%%%%%%%%%%%%%%%%%%%%%%%%%%%%%%%%%%%%%

\title{\pubtitle{}}

%%%%%%%%%%%%%%%%%%%%%%%%%%%%%%%%%%%%%%%%%%%%%%%%%%%%%%%%%%%%%%%%%%%%%
%% Meta-data block
%% ---------------
%% Each author should be given as a separate \author command.
%%
%% Corresponding authors should have an e-mail given after the author
%% name as an \email command. Phone and fax numbers can be given
%% using \phone and \fax, respectively; this information is optional.
%%
%% The affiliation of authors is given after the authors; each
%% \affiliation command applies to all preceding authors not already
%% assigned an affiliation.
%%
%% The affiliation takes an option argument for the short name.  This
%% will typically be something like "University of Somewhere".
%%
%% The \altaffiliation macro should be used for new address, etc.
%% On the other hand, \alsoaffiliation is used on a per author basis
%% when authors are associated with multiple institutions.
%%%%%%%%%%%%%%%%%%%%%%%%%%%%%%%%%%%%%%%%%%%%%%%%%%%%%%%%%%%%%%%%%%%%%

\author{\pubauthA{}}
\affiliation{\pubaddrA{}}
\altaffiliation{Contributed equally to this work}
\author{\pubauthB{}}
\affiliation{\pubaddrA{}}
\alsoaffiliation{\pubaddrB{}}
\altaffiliation{Contributed equally to this work}

\email{\pubemail}

%%%%%%%%%%%%%%%%%%%%%%%%%%%%%%%%%%%%%%%%%%%%%%%%%%%%%%%%%%%%%%%%%%%%%
%% Keywords
%% --------
%%%%%%%%%%%%%%%%%%%%%%%%%%%%%%%%%%%%%%%%%%%%%%%%%%%%%%%%%%%%%%%%%%%%%

% \newcommand{\sep}[0]{}
% \keywords{\pubkeywords{}}