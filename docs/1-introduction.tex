% !TEX root = ../manuscript.tex

%%%%%%%%%%%%%%%%%%%%%%%%%%%%%%%%%%%%%%%%%%%%%%%%%%%%%%%%%%%%%%%%%%%%%
%% Start the main part of the manuscript here.
%%%%%%%%%%%%%%%%%%%%%%%%%%%%%%%%%%%%%%%%%%%%%%%%%%%%%%%%%%%%%%%%%%%%%

\section{Introduction}

Herein we refer to a table (\cref{tbl:example-table}), but also to a figure
(\cref{fig:caption-1}) or a latter equation (\cref{eqn:example}). Finally,
figures (\cref{fig:caption-si}) from the SI can also be referenced. We can add
citations as well \citep{example}. Units are inserted with the help of
\texttt{siunitx}. We can have some standard data \SI{40}{\kilo\joule\per\mol} or
ranges such as \SIrange{20}{30}{\angstrom}. Finally simple unit typesetting is
also possible \si{\mega\hertz\per\kilo\pascal}. 

Chemistry is included by referring to the \texttt{mhchem} package. Simple
molecules like \ce{N2} and \ce{C2H4} should be easy to include. More complex
formula typesetting is possible too: \ce{^{13}C} NMR, \ce{CaCl2 * 12H2O} and
\ce{Fe^{II}Fe^{III}2O4}.

Equations are in a standard Latex \texttt{equation} environment.

\begin{equation}\label{eqn:example}
    e^{i\pi} + 1 = 0
\end{equation}

\lipsum[1-4]{}